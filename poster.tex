\documentclass[landscape,final,a0paper]{baposter}


\tracingstats=2

\usepackage{url}
\usepackage{calc}
\usepackage{graphicx}
\usepackage[numbers]{natbib}
\usepackage{amsmath, amsthm}
\usepackage{enumitem}
\usepackage{amssymb}
\usepackage{pifont}
\usepackage{relsize}
\usepackage{wrapfig}
\usepackage{float}
\usepackage{pgfplots}
\usepackage{tikz}
\usepackage{multirow}
\usepackage{bm}
\usepackage[most]{tcolorbox}
\usepackage{graphicx}
\usepackage{multicol}
\newtheorem{theorem}{Theorem}
\newtheorem{corollary}{Corollary}
\newtheorem{lemma}{Lemma}
\newtheorem{definition}{Definition}
\newtheorem{proposition}{Proposition}
\usepackage{pgfbaselayers}
\pgfdeclarelayer{background}
\pgfdeclarelayer{foreground}
\pgfsetlayers{background,main,foreground}
\usepackage{titlesec}
\titleformat{\section}
  {\normalfont\fontsize{10}{8}\bfseries}{\thesection}{1em}{} %changes font size of section
\usepackage{times}
\usepackage{helvet}
\usepackage{palatino}
\usepackage{caption}
%\newtcolorbox{Definition}{
%colback=green!30!gray!15,
%sharp corners
%}
\newtcolorbox{Definition}{
breakable,
enhanced,
boxrule=0pt,frame hidden,
borderline west={2pt}{0pt}{green!50!black},
colback=green!30!gray!15,
sharp corners,
top = 0.2pt, 
left = 0.2pt, 
bottom = 0.2pt,
right = 0pt
}
\graphicspath{{"/Users/cm/Documents/LaTeX/images/"}}
\selectcolormodel{cmyk}

%\graphicspath{{images/}}

%%%%%%%%%%%%%%%%%%%%%%%%%%%%%%%%%%%%%%%%%%%%%%%%%%%%%%%%%%%%%%%%%%%%%%%%%%%%%%%%
%%%% Some math symbols used in the text
%%%%%%%%%%%%%%%%%%%%%%%%%%%%%%%%%%%%%%%%%%%%%%%%%%%%%%%%%%%%%%%%%%%%%%%%%%%%%%%%
% Format 

\renewcommand{\Pr}{\mbox{P}}
\newcommand{\e}{\mbox{e}}
\newcommand{\dx}{\,\mbox{d}x}

%%%%%%%%%%%%%%%%%%%%%%%%%%%%%%%%%%%%%%%%%%%%%%%%%%%%%%%%%%%%%%%%%%%%%%%%%%%%%%%%
% Multicol Settings
%%%%%%%%%%%%%%%%%%%%%%%%%%%%%%%%%%%%%%%%%%%%%%%%%%%%%%%%%%%%%%%%%%%%%%%%%%%%%%%%
\setlength{\columnsep}{0.7em}
\setlength{\columnseprule}{0mm}
\setlength{\bibsep}{0pt plus 0.3ex}

%%%%%%%%%%%%%%%%%%%%%%%%%%%%%%%%%%%%%%%%%%%%%%%%%%%%%%%%%%%%%%%%%%%%%%%%%%%%%%%%
% Save space in lists. Use this after the opening of the list
%%%%%%%%%%%%%%%%%%%%%%%%%%%%%%%%%%%%%%%%%%%%%%%%%%%%%%%%%%%%%%%%%%%%%%%%%%%%%%%%
\newcommand{\compresslist}{%
\setlength{\itemsep}{1pt}%
\setlength{\parskip}{0pt}%
\setlength{\parsep}{0pt}%
}


%%%%%%%%%%%%%%%%%%%%%%%%%%%%%%%%%%%%%%%%%%%%%%%%%%%%%%%%%%%%%%%%%%%%%%%%%%%%%%
%%% Begin of Document
%%%%%%%%%%%%%%%%%%%%%%%%%%%%%%%%%%%%%%%%%%%%%%%%%%%%%%%%%%%%%%%%%%%%%%%%%%%%%%

\begin{document}

%%%%%%%%%%%%%%%%%%%%%%%%%%%%%%%%%%%%%%%%%%%%%%%%%%%%%%%%%%%%%%%%%%%%%%%%%%%%%%
%%% Here starts the poster
%%%---------------------------------------------------------------------------
%%% Format it to your taste with the options
%%%%%%%%%%%%%%%%%%%%%%%%%%%%%%%%%%%%%%%%%%%%%%%%%%%%%%%%%%%%%%%%%%%%%%%%%%%%%%
% Define some colors
\definecolor{silver}{cmyk}{0,0,0,0.3}
\definecolor{yellow}{cmyk}{0,0,0.9,0.0}
\definecolor{reddishyellow}{cmyk}{0,0.22,1.0,0.0}
\definecolor{black}{cmyk}{0,0,0.0,1.0}
\definecolor{darkYellow}{cmyk}{0,0,1.0,0.5}
\definecolor{darkSilver}{cmyk}{0,0,0,0.1}
\definecolor{lightyellow}{cmyk}{0,0,0.3,0.0}
\definecolor{lighteryellow}{cmyk}{0,0,0.1,0.0}
\definecolor{lighteryellow}{cmyk}{0,0,0.1,0.0}
\definecolor{lightestyellow}{cmyk}{0,0,0.05,0.0}
\definecolor{cyan}{cmyk}{1,0,0,0}
\definecolor{lightcyan}{cmyk}{0.5,0,0,0}
\definecolor{pastelcyan}{cmyk}{0.25,0,0,0}
\definecolor{magenta}{cmyk}{0,1,0,0}
\definecolor{yellow}{cmyk}{0,0,1,0}
\definecolor{lightyellow}{cmyk}{0,0,0.5,0}
\definecolor{pastelyellow}{cmyk}{0,0,0.25,0}
\definecolor{black}{cmyk}{0,0,0,1}
\definecolor{darkgray}{cmyk}{0,0,0,0.75}
\definecolor{gray}{cmyk}{0,0,0,0.5}
\definecolor{lightgray}{cmyk}{0,0,0,0.25}
\definecolor{white}{cmyk}{0,0,0,0}
\definecolor{red}{cmyk}{0,1,1,0}
\definecolor{pastelred}{cmyk}{0,0.25,0,0}
\definecolor{orange}{cmyk}{0,0.5,1,0}
\definecolor{scarlet}{cmyk}{0,1,0.5,0}
\definecolor{brown}{cmyk}{0.5,0.75,1,0}
\definecolor{camel}{cmyk}{0.25,0.375,0.5,0}
\definecolor{cream}{cmyk}{0,0.2,0.3,0}
\definecolor{green}{cmyk}{1,0,1,0}
\definecolor{lightgreen}{cmyk}{0.5,0,0.5,0}
\definecolor{pastelgreen}{cmyk}{0.25,0,0.25,0}
\definecolor{mossgreen}{cmyk}{0.64,0.4,1,0}
\definecolor{yellowgreen}{cmyk}{0.5,0,1,0}
\definecolor{skyblue}{cmyk}{0.6,0.16,0,0}
\definecolor{royal}{cmyk}{1.0,0.5,0,0}
\definecolor{navyblue}{cmyk}{0.9,0.75,0.5,0}
\definecolor{lightnavy}{cmyk}{0.4,0.3,0.2,0}
%\definecolor{myblue}{cmyk}{0.82,0.67,0,0}
%\definecolor{myblue}{cmyk}{1,0,0.2,0.6}
%\definecolor{myblue}{cmyk}{1,0.3,0.2,0}
%\definecolor{myblue}{cmyk}{1,0,0.2,0}
%\definecolor{myblue}{cmyk}{0.55,0.25,0,0}
%\definecolor{myblue}{cmyk}{0,0,0,1}
%\definecolor{myblue}{cmyk}{1,0.61,0,0.45}
\definecolor{myblue}{cmyk}{1,1,0,0}
\definecolor{lightblue}{cmyk}{0.5,0.5,0,0}
\definecolor{pastelblue}{cmyk}{0.25,0.25,0,0}
\definecolor{lightpastelblue}{cmyk}{0.15,0.15,0,0}
\definecolor{lightestpastelblue}{cmyk}{0.05,0.05,0,0}
\definecolor{lavender}{cmyk}{0.25,0.25,0,0}
\definecolor{violet}{cmyk}{0.75,1,0.25,0}
\definecolor{purple}{cmyk}{0.5,1,0.5,0}
\definecolor{lightpurple}{cmyk}{0.25,0.5,0.25,0}
\definecolor{pink}{cmyk}{0,0.5,0,0}
\definecolor{peipei}{cmyk}{27,2,0,0}

%%

\typeout{Poster Starts}
%\background{
  %\begin{tikzpicture}[remember picture,overlay]%
  %  \draw (current page.north west)+(-2em,-2em) node[anchor=north west] %{\hspace{-2em}\includegraphics[height=1.1\textheight]{silhouettes_background}};
 % \end{tikzpicture}%
%}




\newlength{\leftimgwidth}
\begin{poster}%
  % Poster Options, such as colours etc
  {
  % Show grid to help with alignment
  grid=false,
 % Column spacing
  colspacing=0.5em,
 % Color style
 % bgColorOne=pastelblue,
 %bgColorTwo=lightpastelblue,
  bgColorOne=white,
  bgColorTwo=white,
  borderColor=black,
  headerColorOne=pastelblue,
  headerColorTwo=red,
  headerFontColor=black,
 % boxColorOne=lightpastelblue,
 % boxColorTwo=lightestpastelblue,
 boxColorOne=white,
 boxColorTwo=white,
 % Format of textbox
  textborder=roundedleft,
% textborder=rectangle,
% Format of text header
  eyecatcher=true,
  headerborder=open,
  headerheight=0.08\textheight,
  headershape=roundedright,
  headershade=plain,
  headerfont=\large, %Sans Serif
  boxshade=plain,
%  background=shade-tb,
 % background=plain,
  background=none,
  linewidth=1pt
  }
  % Eye Catcher
  {\includegraphics[width=6em, height = 6em]{CryptographyQR.png}} % select eyecatcher=false above if not required. If no eye catcher is present, the title is left aligned.
  % Title
  { %Sans Serif
  %\bf% Serif
  Cracking the Elliptic Curve Cryptosystem} 
  % Authors
  { %Sans Serif
  % Serif
  \vspace{0.5em} 
  Chun Min Tan (CID: 02016068) \\ 
  \url{cmt122@ic.ac.uk}
  }
  % University logo
  { % The makebox allows the title to flow into the logo
    \makebox[8em][r]{%
        \begin{minipage}{16em}
				\hfill \includegraphics[height=5.3em]{imperial.png}
				\end{minipage}
    }
  }

  \tikzstyle{light shaded}=[top color=baposterBGtwo!30!white,bottom color=baposterBGone!30!white,shading=axis,shading angle=30]

  % Width of left inset image
     \setlength{\leftimgwidth}{0.78em+8.0em}

%%%%%%%%%%%%%%%%%%%%%%%%%%%%%%%%%%%%%%%%%%%%%%%%%%%%%%%%%%%%%%%%%%%%%%%%%%%%%%
%%% Now define the boxes that make up the poster
%%%---------------------------------------------------------------------------
%%% Each box has a name and can be placed absolutely or relatively.
%%% The only inconvenience is that you can only specify a relative position 
%%% towards an already declared box. So if you have a box attached to the 
%%% bottom, one to the top and a third one which should be in between, you 
%%% have to specify the top and bottom boxes before you specify the middle 
%%% box.
%%%%%%%%%%%%%%%%%%%%%%%%%%%%%%%%%%%%%%%%%%%%%%%%%%%%%%%%%%%%%%%%%%%%%%%%%%%%%%
    %
    % A coloured circle useful as a bullet with an adjustably strong filling
\newcommand{\colouredcircle}[1]{%
      \tikz{\useasboundingbox (-0.2em,-0.32em) rectangle(0.2em,0.32em); \draw[draw=black,fill=baposterBGone!80!black!#1!white,line width=0.03em] (0,0) circle(0.18em);}}
\small
\setlength{\belowdisplayskip}{0pt} \setlength{\belowdisplayshortskip}{0pt}
\setlength{\abovedisplayskip}{0pt} \setlength{\abovedisplayshortskip}{0pt}%%%%%%%%%%%%%%%%%%%%%%%%%%%%%%%%%%%%%%%%%%%%%%%%%%%%%%%%%%%%%%%%%%%%%%%%%%%%%

  \headerbox{Introduction to Elliptic Curve}{name=intro,column=0,row=0}{
%%%%%%%%%%%%%%%%%%%%%%%%%%%%%%%%%%%%%%%%%%%%%%%%%%%%%%%%%%%%%%%%%%%%%%%%%%%%%%
        \begin{Definition}
        \begin{definition}
    \textbf{(Elliptic Curve)} An elliptic curve $ {\color{myblue}{E}} $ is the set of solutions to a Weierstrass equation
    $$  {\color{myblue}{E : y^2 = x^3 + a x +b}}  $$
     with point $ {\color{myblue}{\mathcal{O}}} $ and $ {\color{myblue}{a,b}} $ satisfying $ {\color{myblue}{4a^3 + 27b^2 \neq 0}}  $
        \end{definition}
        \end{Definition}
    \vspace{-0.5em}
   \textbf{Elliptic Curve Addition Algorithm}  \\
    \vspace{-0.5em}
    \begin{minipage}[c]{5cm}
       Let $ {\color{myblue}{E}} $  be an elliptic curve, and suppose ${\color{myblue}{P, Q}}$ are points on $ {\color{myblue}{E}} $. Define $ {\color{myblue}{P \oplus Q = R'}}  $, as in the figure \cite[p.281]{hoffstein2008introduction}\label{e_add}.
    \end{minipage}%
    \begin{minipage}[c]{6cm}
    \includegraphics[width=0.5\textwidth, height = 0.33\textwidth]{elliptic addition.jpg}
    \end{minipage}  
    
    In cryptography we use elliptic curve over a finite field 
    $$ {\color{myblue}{E(\mathbb{F}_p) = \{(x,y) : x, y \in \mathbb{F}_p \land (x,y)\in E\} \cup \{\mathcal{O}\}}} $$
 }

%%%%%%%%%%%%%%%%%%%%%%%%%%%%%%%%%%%%%%%%%%%%%%%%%%%%%%%%%%%%%%%%%%%%%%%%%%%%%%
  \headerbox{Elliptic Curve Cryptography}{name=ECC,column=0,below=intro}{
  \begin{minipage}[c]{3cm}
  Some notable elliptic curve cryptosystems are Diffie-Hellman key exchange and ElGamal public key, etc. A summary of Diffie-Hellman key exhcange \cite[p.297]{hoffstein2008introduction} 
  \end{minipage}%
  \begin{minipage}[c]{6.1cm}
    \includegraphics[width=0.8\textwidth, height = 0.6 \textwidth]{diffie-hellman-elliptic}
  \end{minipage}
  Note that if one can solve $ {\color{myblue}{Q_A = n_A P}} $ or $ {\color{myblue}{Q_B = n_B P}} $, then one is able to crack the cipher, i.e. the elliptic curve discrete logarithm problem (ECDLP). 
  }

%%%%%%%%%%%%%%%%%%%%%%%%%%%%%%%%%%%%%%%%%%%%%%%%%%%%%%%%%%%%%%%%%%%%%%%%%%%%%%
  \headerbox{Double-and-Add Algorithm}{name=DoubleAdd,column=0,below=ECC}{
%%%%%%%%%%%%%%%%%%%%%%%%%%%%%%%%%%%%%%%%%%%%%%%%%%%%%%%%%%%%%%%%%%%%%%%%%%%%%%
\begin{enumerate}[noitemsep,nolistsep, leftmargin = *]
    \item We write $ {\color{myblue}{n}} $ in binary form as 
    $$ {\color{myblue}{n = n_0 + n_1 \cdot 2 + n_2 \cdot 2^2 + \cdots n_r \cdot 2^r\qquad n_i \in \{0,1\}}}  $$ 
    \item Compute ${\color{myblue}{Q_i = 2 Q_{i-1}=  2^i P}}  $ for $ {\color{myblue}{i\geq 0}}  $. 
    Then 
    \vspace{0.3em}
    \begin{align*}
    {\color{myblue}{nP = n_0 Q_0 + n_1 Q_1 + \cdots + n_r Q_r \qquad n_i \in \{0,1\}}} 
    \end{align*}
    \vspace{-0.9em}
\end{enumerate} 
$ {\color{myblue}{r = \left\lfloor \log_2 n  \right\rfloor \leq \log_2 n}}  $, $ \therefore $ at most $ {\color{myblue}{\boxed{2 \log_2 n }}} $ steps and on average it takes $ {\color{myblue}{\boxed{3 /2 \log_2 n }}} $. \\
  \textbf{Improvements}: We allow coefficients $ {\color{myblue}{n_i \in \{-1, 0 ,1\}}} $ and an extra digit in the expansion.
    \vspace{-0.5em}
    \begin{tcolorbox}[breakable, top  = 0pt, left = 0pt, right= 0pt, bottom =0pt, boxrule = 1pt]
    \begin{proposition}
      For all $ n\in \mathbb{N} $, there exists a ternary expansion where at most half of the coefficients are nonzero. 
    \end{proposition}
    \end{tcolorbox}
    \vspace{-6pt}
    \begin{tcolorbox}[breakable,colback=blue!5!white, colframe=blue!50!black, top=0pt, left=0pt,right=0pt, bottom = 0pt, boxrule = 1pt]
    \begin{proof}
    We look for the first two or more consecutive nonzero $ u_i $ in binary expansion. Suppose we have 
    \vspace{0.3em}
    \begin{align*}
      {\color{myblue}{u_s = u_{s+1} = \cdots = u_{s+t-1} = 1\quad \text{and} \quad u_{s+t} = 0}} 
    \end{align*}
    \vspace{-0.9em}

    where $ {\color{myblue}{t \geq 2}} $. Then we have 
    \vspace{0.5em}
    {\color{myblue}{\begin{align*}
         2^s + 2^{s+1}  + & \cdots +2^{s+t-1}+ 0\cdot 2^{s+t} =  -2^s + 2^{s+t} 
    \end{align*}}}
    \end{proof}
    \end{tcolorbox}
    \vspace{-0.5em}
    There are at most $ {\color{myblue}{\left\lfloor \log_2 n  \right\rfloor  +1}} $ doublings and at most $ {\color{myblue}{\left\lfloor (\left\lfloor \log_2 n  \right\rfloor  + 1 ) / 2 \right\rfloor + 1}}   $ additions, added together gives $ {\color{myblue}{\boxed{3 / 2 \log_2 n + 5/2}}} $ and on average it takes $ {\color{myblue}{\boxed{4 /3 \log_2 n + 7 /3}}} $.
  }  

%%%%%%%%%%%%%%%%%%%%%%%%%%%%%%%%%%%%%%%%%%%%%%%%%%%%%%%%%%%%%%%%%%%%%%%%%%%%%%
 

%%%%%%%%%%%%%%%%%%%%%%%%%%%%%%%%%%%%%%%%%%%%%%%%%%%%%%%%%%%%%%%%%%%%%%%%%%%%%%
  \headerbox{Naive Collision Algorithm}{name=naive,column=1,span=2, row=0, headerColorOne = pastelgreen}{
    \vspace{-0.6em}
    \begin{tcolorbox}[top = 0pt, left=0pt, right = 0pt, bottom = 0pt, boxrule = 1pt]
    \begin{theorem}
    \textbf{(Collision Theorem)} An urn contains $ {\color{myblue}{N}} $ balls, $ {\color{myblue}{n}} $ are red, $ {\color{myblue}{N-n}} $ are blue. Bob chooses $ {\color{myblue}{m}} $ balls with replacement, then if $ {\color{myblue}{X}} $ is the number of red ball observe, we have 
    $ {\color{myblue}{P(X \geq 1 ) \geq 1- e^{- mn / N }}} $
    \end{theorem}
    \end{tcolorbox}
    \vspace{-6pt}
    \begin{tcolorbox}[breakable,colback=blue!5!white, colframe=blue!50!black, top = 0pt, bottom = 0pt, right = 0pt, left = 0pt, boxrule = 1pt]
    \textit{Proof.} 
   Note that $ {\color{myblue}{P(X \geq 1) = 1 - P(X= 0)}} $, note that 
   $ {\color{myblue}{P(X = 0 ) = \left((N-n) / m \right)^m = \left(1- n / N \right)^m \leq  e^{- mn / N}}}  \qquad \square$  
   \vspace{-0.2em}
    \end{tcolorbox}
    \vspace{-0.5em}
    \begin{enumerate}[noitemsep,nolistsep, leftmargin = *]
        \item If $ {\color{myblue}{N = \text{ord}(P)}} $, choose $ {\color{myblue}{r \approx 3 \sqrt{N}}} $. We randomly choose $ {\color{myblue}{1 \leq y_1 , \dots, y_r \leq N}} $ and compute 
        \vspace{0.4em}
        \begin{align*}
        {\color{myblue}{y_1 P , y_2 P , \dots, y_rP \in \langle P \rangle \subseteq E(\mathbb{F}_p) \qquad \text{in at most } 2r \log_2 N \text{ steps}}} 
        \end{align*}
        \vspace{-0.6em}

        \item We randomly select $ {\color{myblue}{1 \leq z_1 , \dots, z_r \leq N}}  $ and compute 
        \vspace{0.4em}
        \begin{align*}
          {\color{myblue}{z_1 P + Q, z_2 P+ Q, \dots, z_rP +Q \in \langle P \rangle \subseteq E(\mathbb{F}_p) \qquad \text{in at most } 2r\log_2 N + r \text{ steps }}}
        \end{align*}
        \vspace{-0.6em}

        \item Search for a collision, $ {\color{myblue}{ y_\alpha P = z_\beta P  + Q \implies Q = (y_\alpha - z_\beta)P}} $, hence the solution is $ {\color{myblue}{y_\alpha - z_\beta \pmod N}}  $. Merge sort plus binary search combined requires on average $ {\color{myblue}{2r\log_2 r}} $ steps. 
    \end{enumerate}
   \textbf{How likely is a collision?}  Treat $ {\color{myblue}{\langle P \rangle}} $ as the urn, $ {\color{myblue}{y_i}} $ as the red balls. Pick $ {\color{myblue}{r}} $ balls $ {\color{myblue}{(z_i P + Q)}} $, then by Collision theorem
        \vspace{0.4em}
   $$ {\color{myblue}{P(\text{at least one collision}) = 1 - \left(1 - r / N\right)^r \geq 1 - e^{-r^2 /N} \approx 1 - e^{-9} \approx 99.98\%}}$$
   with total number of steps 
   $ {\color{myblue}{2r \log_2 N + 2r \log_2 N + r +2r \log_2 r = 3 \sqrt{N} \log_2 (N^4 \cdot 9N) + 3 \sqrt{N} = \boxed{\footnotesize O(\sqrt{N}\log N)}}}$
    \vspace{-0.2em}
  \begin{tcolorbox}[colback = skyblue, boxrule = 0pt, top = 0pt, bottom =0 pt, right= 0 pt, left = 0pt, sharp corners, opacityfill= 0.1]
  \textbf{\textit{Key Idea}}: Generate $ {\color{myblue}{2}} $ lists $ {\color{myblue}{y_iP}}  $ and $ {\color{myblue}{z_iP +Q}} $ by randomly selecting $ {\color{myblue}{y_i}} $ and $ {\color{myblue}{z_i}} $ from $ {\color{myblue}{[1 ,  N]}} $ and search for a collision. 
  \end{tcolorbox}
  \vspace{-0.5em}
 }

%%%%%%%%%%%%%%%%%%%%%%%%%%%%%%%%%%%%%%%%%%%%%%%%%%%%%%%%%%%%%%%%%%%%%%%%%%%%%%
\headerbox{Pollard's $ \rho $ method for $ ord(P) = p $ prime}{name=rho,column=1,span=2, below = naive, headerColorOne = pastelgreen}{
  \vspace{-0.6em}
  \begin{tcolorbox}[top = 0pt, left=0pt, right = 0pt, bottom = 0pt, boxrule = 1pt]
  \begin{theorem}
  Set $ {\color{myblue}{|S| = p}} $, $ {\color{myblue}{f : S \to S}} $ is sufficiently random, if $ {\color{myblue}{(x_i)}} $ is generated by applying $ {\color{myblue}{f}} $, then $ {\color{myblue}{\text{E}(T+M)  = \sqrt{ \pi p / 2}}}$.
  \end{theorem}
  \end{tcolorbox}
    \vspace{-0.4em}
  \begin{minipage}[c]{12.5cm}
    %\vspace{-2.6em}
  \begin{tcolorbox}[breakable,colback=blue!5!white, colframe=blue!50!black, top = 0pt, left=0pt, right = 0pt, bottom = 0pt, boxrule = 1pt]
  \begin{proof}
 \footnotesize One can show that for large $ {\color{myblue}{p}} $ and $ {\color{myblue}{T,M}} $ as in the figure provided \cite[p.235]{hoffstein2008introduction}
\vspace{-1em}
 $$ {\footnotesize{{\color{myblue}{P( \substack{x_0 , x_1 , \dots, x_{k-1} \\ \text{are all different}}) = \prod\limits_{i = 1}^{k-1} P (\substack{x_i \neq x_j \text{for} \\ \text{all } 0\leq j < i} | \substack{x_0 ,x_1 ,\dots, x_{i-1} \\ \text{are all different}}) = \prod\limits_{i = 1}^{k-1} (1 - i / p) \approx e^{- k^2 / 2p}}}} \vspace{-5em}}  $$ 
 \vspace{-1em}

 then $ {\color{myblue}{E(T+M) = \sum\limits_{k =1 }^{\infty} k P(x_k\text{ is first match})\approx \sum\limits_{k =1}^{\infty} (k^2 / p ) e^{- k^2 / 2p} = \sqrt{\pi p / 2}}}   $
  \end{proof}
  \end{tcolorbox}
  \end{minipage}%
  \begin{minipage}[c]{4cm}
    \vskip -1.5em
\begin{figure}[H]
    \includegraphics[width=1\textwidth, height = 0.5\textwidth]{rho_method1.jpeg}
\end{figure}
  \end{minipage}
  \begin{enumerate}[noitemsep,nolistsep, leftmargin = *]
    \item Let $ {\color{myblue}{\{S_1 , \dots, S_L\}}} $ be a partition of $ {\color{myblue}{\langle P \rangle}} $. Now define $ {\color{myblue}{H(X) =  j}}  $  if $ {\color{myblue}{X \in S_j}}  $ and let $ {\color{myblue}{a_j, b_j \in_R [0,p-1]}}$ for each $ {\color{myblue}{1 \leq j \leq L}}  $. Then define 
    \vspace{-2em}
    \small
    {\color{myblue}{\begin{align*}
        f &: \langle P \rangle \to \langle P \rangle \\ 
            &\quad  X \mapsto X + a_j P + b_j Q \qquad \text{where } j = H(X)
    \end{align*}}}
    \vspace{-2em}
    \item Since $ {\color{myblue}{\langle P \rangle}} $ is finite, by \textbf{\textit{Floyd's cycle-detection algorithm}}, compute $ {\color{myblue}{(X_i, X_{2i})}} $ until $ {\color{myblue}{X_i = X_{2i}}} $, where $ {\color{myblue}{X_i = f(X_{i-1})}} $. 
  \end{enumerate}
  \textbf{How fast is this algorithm?} Since $ {\color{myblue}{\text{E}(T+M) = \sqrt{p \pi / 2}}} $, and in each evaluation of the function $ {\color{myblue}{f}} $, it only requires $ {\color{myblue}{2}} $ modular addition, hence it takes approximately $ {\color{myblue}{2 \cdot \sqrt{p \pi / 2}}} $ steps, therefore $ {\color{myblue}{\footnotesize\boxed{O(\sqrt{p \pi /2})}}} $ or $ {\color{myblue}{\boxed{O(\sqrt{p})}}} $ . 
  \vspace{-0.2em}
  \begin{tcolorbox}[colback = skyblue, boxrule = 0pt, top = 0pt, bottom =0 pt, right= 0 pt, left = 0pt, sharp corners, opacityfill= 0.1]
  \textbf{\textit{Key Idea}}: Construct $ {\color{myblue}{f}} $ to be sufficiently random, then compute $ {\color{myblue}{(X_i, X_{2i})}} $ until $ {\color{myblue}{X_i = X_{2i}}} $ where $ {\color{myblue}{X_i = f(X_{i-1})}} $. 
  \end{tcolorbox}
  \vspace{-0.5em}
  }
%%%%%%%%%%%%%%%%%%%%%%%%%%%%%%%%%%%%%%%%%%%%%%%%%%%%%%%%%%%%%%%%%%%%%%%%%%%%%%
\headerbox{Pohlig-Hellman Algorithm}{name=PH,column=1, span = 2, below = rho, headerColorOne = pastelgreen}{
  Let $ {\color{myblue}{N = \text{ord}(P)}}  {\color{myblue}{ = p_1 ^{e_1 } p_2 ^{e_2 } \cdots p_r^{e_r}}} $ be given. To solve $ {\color{myblue}{\tilde{n} : Q = \tilde{n} P}} $, consider $ {\color{myblue}{\tilde{n} \equiv n_i \pmod {p_i^{e_i}}}} $ for each $ {\color{myblue}{1 \leq i \leq r}} $. 
  \begin{enumerate}[noitemsep,nolistsep, leftmargin = *]
    \item To find $ {\color{myblue}{n_i \pmod {p_i^{e_i}}}} $, we write it in its $ {\color{myblue}{p_i}} $-adic expansion modulo $ {\color{myblue}{p_i^{e_i}}} $. This will give 
    \vspace{0.3em}
    \begin{align*}
      {\color{myblue}{n_i \equiv z_0 + z_1 p_i + \cdots + z_{e_i - 1}p_i^{e_i -1 } \pmod{p_i^{e_i}}  \qquad \text{ where } z_i \in [0,p_i -1 ]}}
    \end{align*}
    \vspace{-0.9em}
    \item Define $ {\color{myblue}{P_0 = (N / p_i ) P}} $ and $ {\color{myblue}{Q_0 = (N / p _ i) Q}} $, since $ {\color{myblue}{P_0}}  $ has order $ {\color{myblue}{p_i}} $, we have 
    \vspace{0.3em}
    \begin{align*}
      {\color{myblue}{Q_0 = (N / p_i)  Q = (N/p_i) \tilde{n } P = \tilde{n} ( N / p_i \cdot  P) = \tilde{n} P_0  = [\tilde{n}]_{p_i} P_0 = z_0 P_0}}   
    \end{align*}
    \vspace{-0.9em} 

    therefore $ {\color{myblue}{z_0 = \log_{P_0 } Q_0}}  $ which can be solved using Pollard's $ {\color{myblue}{\rho}} $ method. 
    \item If $ {\color{myblue}{z_0 , \dots, z_{t-1}}} $ have been computed, then $ {\color{myblue}{z_t = \log_{P_0 } Q_t}} $  can be computed using Pollard's $ {\color{myblue}{\rho}} $ method as well where 
    $$ {\color{myblue}{Q_t = \frac{N}{p_i^{t+ 1}} (Q - z_0 P - z_1 p_i P  - \cdots - z_{t-1}p_i^{t-1})}} $$
    \item This allow us to compute $ {\color{myblue}{z_0 , \dots, z_{e_i -1}}} $. Repeat for all $ {\color{myblue}{n_j}}  $, then $ {\color{myblue}{\tilde{n}}} $ can be found using \textbf{\textit{Chinese Remainder Theorem}}. 
  \end{enumerate}
 \textbf{How fast is this algorithm?} Given $ {\color{myblue}{N = p_1 ^{e_1 } p_2 ^{e_2 } \cdots p_r^{e_r}}}   $, then Pohlig-Hellman Algorithm takes $ {\color{myblue}{\scriptsize\boxed{ O \left(\sum_{i=1}^{r } e_i (\log n + \sqrt{p_i})\right)}}} $ 
  \begin{tcolorbox}[colback = skyblue, boxrule = 0pt, top = 0pt, bottom =0 pt, right= 0 pt, left = 0pt, sharp corners, opacityfill= 0.1]
  \textbf{\textit{Key Idea}}: Solve for $ {\color{myblue}{\tilde{n} \equiv n\pmod{p_i^{e_i}}}} $ by solving ${\color{myblue}{Q_t = z_t P}}  $ and combine each $ {\color{myblue}{n_i}} $ using Chinese Remainder Theorem.
  \end{tcolorbox}
  \vspace{-0.5em}
  }

%%%%%%%%%%%%%%%%%%%%%%%%%%%%%%%%%%%%%%%%%%%%%%%%%%%%%%%%%%%%%%%%%%%%%%%%%%%%%
  \headerbox{General Attack on ECDLP}{name=general_atk,column=3,row=0}{
  The best general attack is a combination of Pollard's $ {\color{myblue}{\rho}} $ algorithm for factorisation and the Pohlig-Hellman Algorithm. The expected running time of this is 
  \vspace{0.8em}
  $$ {\color{myblue}{\boxed{O(\sqrt{p})}}} $$
  where $ {\color{myblue}{p}} $ is the largest prime divisor of $ {\color{myblue}{N = \text{ord}(P)}} $. 
  \begin{enumerate}[leftmargin = *]
    \item[\ding{43}] To resist this attack one should pick an elliptic curve with point $ {\color{myblue}{P}} $ of order $ {\color{myblue}{N}} $ where all the prime divisors of $ {\color{myblue}{N}} $ are large\cite[\S 4.1]{hankerson2006guide}.
  \end{enumerate}
  
  }


%%%%%%%%%%%%%%%%%%%%%%%%%%%%%%%%%%%%%%%%%%%%%%%%%%%%%%%%%%%%%%%%%%%%%%%%%%%%%%
 \headerbox{Million Dollar Question(?)}{name=continuous,column=3,below=general_atk, headerColorOne = pastelred}{
  \vspace{-1.6em}
\begin{figure}[H]
    \centering
    \includegraphics[width=0.7\textwidth, height = 0.25\textwidth]{PNP.jpg}
    \vspace*{-3mm}
    \caption*{\footnotesize Complexity Classes Venn Diagram \cite{de2014edge}}
\end{figure}
\vspace{-1.5em}
  \begin{Definition}
  \begin{definition}
  \textbf{(Decision version ECDLP)} Given $ {\color{myblue}{(E(\mathbb{F}_p), N, d , Q , P)}} $ where $ {\color{myblue}{E(\mathbb{F}_p)}} $ is an elliptic curve over $ {\color{myblue}{\mathbb{F}_p}} $, $ {\color{myblue}{P \in E(\mathbb{F}_p)}} $ with order $ {\color{myblue}{N}} $ and $ {\color{myblue}{d \leq N}}  $ an integer. Then the decision problem is: 
$$ {\color{myblue}{\text{Is there an integer } k\leq d : Q = k P \text{?}}} $$
  \end{definition}
  \end{Definition}
  The decision version of ECDLP is known to be in $ \text{NP}\cap \text{co-NP} $. It is not known to be in P as currently there is no deterministic polynomial time algorithm that solves it. \textbf{Therefore if one can show that there does not exist a deterministic polynomial time algorithm that solves ECDLP, this would imply}  
  $$ {\color{red}{ \text{P} \neq \text{NP}}} $$
  thus settling one of the most important question in computer science. Further if one can show that it is NP-complete, then this would imply that 
  $$ \text{NP} = \text{co-NP} $$
  which also solves another unsolved question in computer science\cite[\S 4.1]{hankerson2006guide}. 
  }

%%%%%%%%%%%%%%%%%%%%%%%%%%%%%%%%%%%%%%%%%%%%%%%%%%%%%%%%%%%%%%%%%%%%%%%%%%%%%%
\headerbox{Conclusions}{name=concl,column=3,below=continuous}{
 The underlying working principle of ECDLP is based on the assumption that $ \text{P} \neq \text{NP} $. The algorithms covered here involve some probabilistic elements and the fastest known algorithm that can solve ECDLP in polynomial time can only be done on a non-deterministc Turing machine. But all of this may change when the age of quantum computing truly begins. 
	}

%%%%%%%%%%%%%%%%%%%%%%%%%%%%%%%%%%%%%%%%%%%%%%%%%%%%%%%%%%%%%%%%%%%%%%%%%%%%%%
  \headerbox{References}{name=references,column=3, below = concl}{
    \tiny
    \vspace{-0.4em}
    \bibliographystyle{unsrtnat}
    \renewcommand{\section}[2]{\vskip 0.05em}
    \bibliography{references}
    \vspace{-0.01em}
  }
%%%%%%%%%%%%%%%%%%%%%%%%%%%%%%%%%%%%%%%%%%%%%%%%%%%%%%%%%%%%%%%%%%%%%%%%%%%%%%
  
\end{poster}



\end{document} 